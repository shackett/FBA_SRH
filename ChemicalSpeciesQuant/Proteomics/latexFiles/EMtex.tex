\documentclass[12pt]{article}
\usepackage[left=0.5in,top=0.5in,right=0.5in,bottom=0.5in,nohead]{geometry}
\usepackage{geometry}                % See geometry.pdf to learn the layout options. There are lots.
\geometry{a4paper}                   % ... or a4paper or a5paper or ... 
\usepackage{wrapfig}	%in-line figures
\usepackage[numbers, super]{natbib}		%bibliography
%\usepackage{pslatex} 	%for times new roman
\usepackage[parfill]{parskip}    % Activate to begin paragraphs with an empty line rather than an indent
\usepackage{graphicx}
\usepackage{amssymb}
\usepackage{epstopdf}
\usepackage{booktabs}
\usepackage{amsmath}
\usepackage{subfig}
\usepackage{wrapfig}
\usepackage{graphics}
\usepackage{enumitem}
\usepackage[T1]{fontenc}
\usepackage{aurical}
\usepackage[scaled]{helvet}
\usepackage{multicol}
\usepackage{upquote}
\usepackage{tikz}

\DeclareGraphicsRule{.tif}{png}{.png}{`convert #1 `dirname #1`/`basename #1 .tif`.png}

\author{Sean R. Hackett}
\title{Peptide $\rightarrow$ prot EM} 
\date{}

\begin{document}

\setlength{\parskip}{1mm}
\linespread{1}

\thispagestyle{empty}
\pagestyle{empty}

\maketitle

\section*{Going from peptides to proteins via expectation-maximization}

When using relative variation in peptides to predict variation in protein abundance we need to deal with two factors
\begin{itemize}
\item If a peptide maps to multiple proteins, it should be attributed to a protein (thereby adding signal) to the extent that its pattern matches the pattern from other peptides.
\item Some peptides won't conform to the trends of their protein because they may be the non-covalently modified complement of a set of unascertained modified peptides.  These peptides shouldn't inform the general protein trend, and may be interesting to analyze in isolation.
\end{itemize}

\subsection*{Algorithm structure}

\textbf{I} = Peptides, \textbf{C} = Conditions, \textbf{K} = Proteins

\begin{itemize}
\item[\textbf{X}$_{IC}$:] Data matrix: Input MS data of the relative abundance of each peptide across \textbf{C} conditions
\item[$\sigma^{2}_{IC}$/$\tau^{2}_{IC}$:] [I] Fitted variance/precision relative to peptide IC for each spectra.
\item[\textbf{M}$_{IK}$:] Possible mappings between peptides (\textbf{I}) and proteins (\textbf{K})
\item[$\mathbf{\Theta}_{IK}$:] Responsibility matrix: fraction of peptide \textbf{i} contributed by protein \textbf{k}.
\item[$\mathbf{\Omega}_{KC}$:] Point estimate for each protein$\cdot$condition.
\item[$\pi_{I}$:] Peptide \textbf{i} matches a protein ($\pi_{i}$ = 1) or is a divergent peptide ($\pi_{i}$ = 0).
\item[$\alpha_{K}$:] Protein \textbf{k} is supported by the data ($\alpha_k$ = 1) or is subsumable and sufficiently described by trends of other proteins ($\alpha_{k}$ = 0)

\end{itemize}

\begin{align}
Lik(\mathbf{\Omega}, \mathbf{\Theta}, \pi, \alpha | \sigma^{2}, \textbf{X}) &= \prod_{i}^{I}\prod_{c}^{C}\left[ \pi_{i}\mbox{\Large \textbf{N}}(\sum_{k}^{K}\mathbf{\Theta}_{iK}\alpha_{K}\mathbf{\Omega}_{Kc} ; \textbf{X}_{ic}, \sigma^{2}_{ic}) + (1-\pi_{i})\mbox{\Large \textbf{N}}( \textbf{X}_{ic}; \textbf{X}_{ic}, \sigma^{2}_{ic}) \right] \notag\\
&\cdot \prod_{i}^{I}[\pi_{i}p(\pi_{i} = 1) + (1-\pi_{i})p(\pi_{i} = 0)]\notag\\
Lik(\mathbf{\Omega}, \mathbf{\Theta}, \pi, \alpha | \sigma^{2}, \textbf{X}) &= \prod_{i}^{I}\prod_{c}^{C}\left[\pi_{i}\frac{1}{\sigma_{ic}\sqrt{2\pi}}e^{-\frac{1}{2}(\frac{\mathbf{\Theta}_{iK}\alpha_{K}\mathbf{\Omega}_{Kc} - \textbf{X}_{ic}}{\sigma_{ic}})^2} + (1-\pi_{i})\frac{1}{\sigma_{ic}\sqrt{2\pi}}e^{0}\right]\notag\\
&\cdot \prod_{i}^{I}[\pi_{i}p(\pi_{i} = 1) + (1-\pi_{i})p(\pi_{i} = 0)]\notag\\
Lik(\mathbf{\Omega}, \mathbf{\Theta}, \pi, \alpha | \sigma^{2}, \textbf{X}) &= \prod_{i}^{I}\prod_{c}^{C}\left[\frac{1}{\sigma_{ic}\sqrt{2\pi}} \right] \prod_{i}^{I}\prod_{c}^{C}\left[\pi_{i}e^{-\frac{1}{2}(\frac{\mathbf{\Theta}_{iK}\alpha_{K}\mathbf{\Omega}_{Kc} - \textbf{X}_{ic}}{\sigma_{ic}})^2} + (1-\pi_{i})\right]\notag\\
&\cdot \prod_{i}^{I}[\pi_{i}p(\pi_{i} = 1) + (1-\pi_{i})p(\pi_{i} = 0)]\notag\\
\textit{l}(\mathbf{\Omega}, \mathbf{\Theta}, \pi, \alpha | \sigma^{2}, \textbf{X}) &= \text{constant} + \sum_{i}^{I}\sum_{c}^{C}ln\left[\pi_{i}e^{-\frac{1}{2}(\frac{\mathbf{\Theta}_{iK}\alpha_{K}\mathbf{\Omega}_{Kc} - \textbf{X}_{ic}}{\sigma_{ic}})^2} + (1-\pi_{i})\right]\notag\\
&+ \sum_{i}^{I}ln[\pi_{i}p(\pi_{i} = 1) + (1-\pi_{i})p(\pi_{i} = 0)]\notag\\
\end{align}

\subsection*{Updating $\pi_{i}$}

Since $\pi_{i}$ can only take binary values, maximizing the log-likelihood over $\pi_{i}$ amounts to comparing the log-likelihood under these two scenarios

\begin{align}
\pi_{i} &= 1: \textit{logL} = \text{constant} + \sum_{c}^{C}\left[{-\frac{1}{2}(\frac{\mathbf{\Theta}_{iK}\alpha_{K}\mathbf{\Omega}_{Kc} - \textbf{X}_{ic}}{\sigma_{ic}})^2}\right] + ln[p(\pi_{i} = 1)]\notag\\
\pi_{i} &= 0: \textit{logL} = \text{constant} + \sum_{c}^{C}ln\left[1\right] + ln[p(\pi_{i} = 0)]\notag\\
\pi_{i} &= 1 \hspace{2mm}\text{if}:\notag\\ & \sum_{c}^{C}\left[{-\frac{1}{2}(\frac{\mathbf{\Theta}_{iK}\alpha_{K}\mathbf{\Omega}_{Kc} - \textbf{X}_{ic}}{\sigma_{ic}})^2}\right] + ln[p(\pi_{i} = 1)] > ln[p(\pi_{i} = 0)]
\end{align}

\subsection*{Updating $\mathbf{\Theta}_{ik}$}

\begin{align}
\textit{l}(\mathbf{\Theta}_{iK} | \mathbf{\Omega}, \pi, \alpha, \sigma^{2}, \textbf{X}) &= \text{constant} + \sum_{c}^{C}ln\left[\pi_{i}e^{-\frac{1}{2}(\frac{\mathbf{\Theta}_{iK}\alpha_{K}\mathbf{\Omega}_{Kc} - \textbf{X}_{ic}}{\sigma_{ic}})^2} + (1-\pi_{i})\right]\notag\\
\end{align}

Assume $\pi_{i}$ = 1, if $\pi_{i}$ = 0 then the following calculations will maximize the log-likelihood under the model that $\pi_{i}$ = 1 and then scale these by $\pi_{i}$ to zero

\begin{align}
\textit{l}(\mathbf{\Theta}_{iK} | \mathbf{\Omega}, \pi, \alpha, \sigma^{2}, \textbf{X}) &= \text{constant} + \sum_{c}^{C}\left[{-\frac{1}{2}(\frac{\mathbf{\Theta}_{iK}\alpha_{K}\mathbf{\Omega}_{Kc} - \textbf{X}_{ic}}{\sigma_{ic}})^2}\right]\notag\\
 &= \text{constant} -\frac{1}{2}\sum_{c}^{C}\left[{\tau_{ic}^{2}(\mathbf{\Theta}_{iK}\alpha_{K}\mathbf{\Omega}_{Kc} - \textbf{X}_{ic})^2}\right]\notag\\
\end{align}

Maximizing this log-likelihood is equivalent to minimizing the weighted residuals of the above equation with the constraints that $\sum_{k}^{K}\mathbf{\Theta}_{ik} = 1$ and $\mathbf{\Theta}_{ik} \ge 0$
\subsection*{Updating $\mathbf{\Omega}_{kc}$}

The mean and precision of protein pattern inference can be determined by seeing that this is equivalent to a product of normal distributions, for which a closed-form solution exists through \textit{integrated likelihood}.

$\tau_{y} = \prod_{z \neq y}^{Y}\sigma^{2}_{z}$

\begin{align}
\mathbf{\Omega}_{kc} \sim \mbox{\Large \textbf{N}}\left(\mu = \frac{\sum_{i = 1}^{I}\mathbf{\Theta}_{ik}\alpha_{k}\textbf{X}_{ic}\tau_{i}}{\sum_{i = 1}^{I}\mathbf{\Theta}_{ik}\alpha_{k}\tau_{i}}, \sigma^{2} =  \left(\sum_{i = 1}^{I}\frac{\mathbf{\Theta}_{ik}\alpha_{k}}{\sigma^{2}_{i}}\right)^{-1} \right)
\end{align}

\subsection*{Updating $\mathbf{\alpha}_{k}$}

When protein \textbf{k} is subsumable, i.e. the peptides that match it can all be matched to other proteins as well, we want to consider whether sufficient evidence of a unique trend in \textbf{k} exists before we use it.  To maximize the log-likelihood over the binary possible values of $\alpha_{k}$, we can consider a reduced log-likelihood:

\begin{align}
\textit{l}(\alpha_{k} | \mathbf{\Omega}, \mathbf{\Theta}, \pi, \sigma^{2}, \textbf{X}) &= \text{constant} + \sum_{i}^{I}\sum_{c}^{C}ln\left[\alpha_{k}e^{-\frac{1}{2}(\frac{\mathbf{\Theta}_{iK}\alpha_{k  = 1}\mathbf{\Omega}_{Kc} - \textbf{X}_{ic}}{\sigma_{ic}})^2} + (1-\alpha_{k})e^{-\frac{1}{2}(\frac{\mathbf{\Theta}_{iK}\alpha_{k  = 0}\mathbf{\Omega}_{Kc} - \textbf{X}_{ic}}{\sigma_{ic}})^2}\right]\notag\\
&+ ln[\alpha_{k}p(\alpha_{k} = 1) + (1-\alpha_{k})p(\alpha_{k} = 0)]\notag
\end{align}

\begin{align}
\alpha_{k} &= 1: \textit{logL} = \text{constant} + \sum_{i}^{I}\sum_{c}^{C}\left[{-\frac{1}{2}(\frac{\mathbf{\Theta}_{iK}\alpha_{k  = 1}\mathbf{\Omega}_{Kc} - \textbf{X}_{ic}}{\sigma_{ic}})^2}\right] + ln[p(\alpha_{k} = 1)]\notag\\
\alpha_{k} &= 0: \textit{logL} = \text{constant} + \sum_{i}^{I}\sum_{c}^{C}\left[{-\frac{1}{2}(\frac{\mathbf{\Theta}_{iK}\alpha_{k  = 0}\mathbf{\Omega}_{Kc} - \textbf{X}_{ic}}{\sigma_{ic}})^2}\right] + ln[p(\alpha_{k} = 0)]\notag\\
\pi_{i} &= 0 \hspace{2mm}\text{if}:\notag\\ \sum_{i}^{I}\sum_{c}^{C}&\left[{-\frac{1}{2}(\frac{\mathbf{\Theta}_{iK}\alpha_{k  = 0}\mathbf{\Omega}_{Kc} - \textbf{X}_{ic}}{\sigma_{ic}})^2}\right] + ln[p(\alpha_{k} = 0)] \ge \sum_{i}^{I}\sum_{c}^{C}\left[{-\frac{1}{2}(\frac{\mathbf{\Theta}_{iK}\alpha_{k  = 1}\mathbf{\Omega}_{Kc} - \textbf{X}_{ic}}{\sigma_{ic}})^2}\right] + ln[p(\alpha_{k} = 1)]
\end{align}

\subsubsection*{Notes}

\begin{itemize}
\item For every protein that only has peptides that are shared by other proteins, there is little evidence that this protein exists unless the shared peptides exhibit a trend that diverges from the behavior of the shared proteins (as determined by its unique peptides).  
\end{itemize}


\end{document}